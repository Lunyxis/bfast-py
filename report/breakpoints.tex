\documentclass[main.tex]{subfiles}

\begin{document}
\chapter{Estimation of Breakpoints}
\label{chap:estimation_of_breakpoints}
In the paper from 2003 \cite{bai_perron}, Jushan Bai and Pierre Perron provide an
efficient algorithm for time series breakpoint estimation. It uses a dynamic-programming
approach and requires $O(T^2)$ least-squares operations for any number of breakpoints.

\section{The Model}
\label{sec:breakpoints_the_model}

\section{Recursive Residuals}
\label{sec:recursive_residuals}

\section{The Triangular Matrix of Sums of Squared Residuals}
\label{sec:triangular_matrix}
\subsection{Reductions}
\subsection{Computation of the triangular matrix}

\section{The dynamic programming algorithm}
\label{sec:dynamic_algorithm}

\section{Bayesian Information Criterion}
\label{sec:bayesian_information_criterion}
\cite{schwarz1978}

\section{Steps of the Algorithm}
\label{sec:steps_of_the_algorithm}
The algorithm by Bai and Perron works by taking following steps:
\begin{enumerate}
\item \textbf{Calculate the upper-triangular matrix of sums of squared residuals}\\
  We calculate the matrix, by following the guidelines, described in \ref{sec:triangular_matrix}.
  For each entry in the matrix, we perform the linear regression using the given the matrix
  $X$ and observations $y$, and the recursive residuals are calculated 
  using \ref{sec:recursive_residuals}.
\item \textbf{Find the largest number of allowed segments}\\
  Given a value of $0\leq h' \leq 1$, which signifies the smallest allowed size of
  a segment, and $m_{\text{user}}$ which is the maximum number of segments, determined by the user, 
  we can then find the largest number of allowed
  segments $m_{\text{max}} = \min(\floor{\frac{1}{h'}}, m_{\text{user}})$.
\item \textbf{Create a table \texttt{BIC} of size $m_{\text{max}}$}
\item \textbf{For $m \in (m_{\text{max}}, ... 1)$, do following:}
  \begin{enumerate}[1)]
  \item Find the optimal position of the $m-1$ breakpoints, using algorithm
    \ref{sec:dynamic_algorithm}.
  \item Calculate the sum of squared residuals, and then use it to
    calculate the Bayesian Information Criterion $\text{BIC}_m$ of this partition, using
    \ref{sec:bayesian_information_criterion}.
  \item Add $\text{BIC}_m$ to the table
  \end{enumerate}
\item \textbf{Find the value of $m_{best}$ that corresponds to the minimal value in the \texttt{BIC}}
\item \textbf{Use the algorithm \ref{sec:dynamic_algorithm} to find the $m_{best} - 1$ breakpoints}
\end{enumerate}

\section{Parameters and Their Values}
\label{sec:breakpoints_params}
The algorithm takes two parameters:
\begin{itemize}
\item  $0\leq h' \leq 1$, which can be used to calculate the minimal allowed size of the segment
  $h = h' n$. The default value is $h' = 0.15$.
\item \texttt{max\_breaks}, which can be used to calculate the maximum number of allowed segments
  $m_{\text{user}} = \texttt{max\_breaks} + 1$. This parameter is optional.
\end{itemize}


%% \subsection{Calculation of confidence intervals for the break dates}
%% \label{subsec:confidence_intervals}

\biblio
\end{document}
