\documentclass[main.tex]{subfiles}

\begin{document}
\chapter{Conclusion and Future Work}
\label{chap:conclusion_and_future_work}
I have completed a working prototype implementation of BFAST and BFAST0n
structural change detection algorithms by Verbesselt et al. \cite{bfast} in
Python. My implementation was validated using the datasets from the original
paper and results were compared to the original R implementation \cite{bfast-github}.

This report chronicles my implementation of the BFAST and BFAST0n and covers the
underlying theory of both algorithms. In particular, I describe in detail the
concrete steps of each component of BFAST. This includes the STL decomposition
algorithm \cite{stl}, OLS-MOSUM test by Achim Zeileis \cite{strucchange}, the
breakpoint estimation algorithm by Bai and Perron \cite{bai_perron} and BFAST
itself. The description of the concrete algorithm steps can be helpful to a
potential programmer that would like to implement one of those algorithms
without diving too deep into the statistical theory. 

However, some parts of the algorithm have been omitted and should be added in
the future. This includes the computation of the confidence intervals for the
breakpoints in the algorithm by Bai and Perron. The underlying theory was too
overwhelming for a person without a background in either statistics or
econometrics and therefore was given the lowest priority, since both BFAST and
BFAST0n can function without it.

Moreover, the differences between the BFAST and BFAST Monitor \cite{bfast_monitor}
approach are not covered in this report either, since it would be a crucial part of my
MSc thesis, and this project is focused around the theory of the BFAST and its components.

Since performance was not the aim of this particular project, this
implementation is rather slow and is comparable to the original R
implementation. It struggles with any time series with more than 500
observation (as demonstrated by the \texttt{NDVI} dataset. In addition, the Python
implementation is about 30\% slower than the original implementation, since
the original uses C++ for some of its most resource-demanding components.

The result of this project would form a solid foundation for
the future work, where a parallelized, GPU-based implementation would be developed and
could potentially serve as an alternative to the parallelized implementation of
BFAST-Monitor. 


\biblio
\end{document}
