\documentclass[main.tex]{subfiles}

\begin{document}
\chapter{Practical Information}
\label{a_chap:practical}
\section*{Source Files}
\label{sec:source_files}
The source code for the implementation, described in this report, can be found
on GitHub \cite{my-github} and is structured in a following manner:
\begin{itemize}
\item \texttt{bfast0n.py}\\
  Implementation of the BFAST0N algorithm, described in Chapter \ref{chap:bfast0n}
\item \texttt{bfast.py}\\
  Implementation of the BFAST algorithm, described in Chapter \ref{chap:bfast}
\item \texttt{breakpoints.py}\\
  Implementation of the breakpoint estimation algorithm, described in Chapter \ref{chap:breakpoints}
\item \texttt{converter.py}\\
  Script that converts the datasets for the R programming language (\texttt{.rda}) into numpy
  data files (\texttt{.npy}). It requires a working R installation.
\item \texttt{datasets.py}\\
  Collection of datasets for testing. They are taken from the R programming language standard library
  \cite{r-datasets} and
  the \texttt{strucchange} package for the R-language \cite{strucchange_code}.
\item \texttt{loess.py}\\
  Implementation of the LOESS algorithm (Chapter \cite{chap:stl}). Used only for demonstration.
\item \texttt{efp.py}\\
  Implementation of the empirical fluctuation processes, in particular the OLS-MOSUM test that is
  described in Chapter \ref{chap:mosum}.
\item \texttt{plots.py}\\
  Script that produces all the plots from the validation chapter (Chapter \ref{chap:validation}).
\item \texttt{recresid.py}\\
  Implementation of the recursive residuals for the linear regression relationships,
  as described in Chapter \ref{chap:breakpoints}.
\item \texttt{ssr\_triang.py}\\
  Calculation of the upper-triangular matrix of sums of squared residuals for
  the breakpoint estimation algorithm as described in Chapter \ref{chap:breakpoints}.
\item \texttt{stl.py} \\
  A wrapper for the \texttt{statsmodels} \cite{statsmodels} implementation of the STL decomposition as
  described in Chapter \ref{chap:stl}.
\item \texttt{utils.py} \\
  Helper functions and lookup tables. Used by other modules. 
\end{itemize}

\section*{Dependencies}
\label{sec:deps}
The implementation was tested using Python 3.8.3.
In order to install all the necessary libraries, start the virtual environment and import
the requirements:
\begin{verbatim}
pip install -f requirements.txt
\end{verbatim}

\section*{How to Run the Tests}
\label{sec:source_code_overview}
Tests for each file in the \emph{src} directory are contained withing that
source file. In order to run test use following:
\begin{verbatim}
python file.py
\end{verbatim}
In order to get a more verbose output, run:
\begin{verbatim}
python file.py --log=INFO
\end{verbatim}
In order to see the debug information, run:
\begin{verbatim}
python file.py --log=DEBUG
\end{verbatim}
In order to reproduce the plots, run:
\begin{verbatim}
python plots.py
\end{verbatim}

\biblio
\end{document}
