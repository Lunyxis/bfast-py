\documentclass[main.tex]{subfiles}

\begin{document}
\section{OLS-MOSUM Test}
\label{sec:mosum}

Tests from the generalized fluctuation test framework \cite{kuan_hornik}
is one of the most important classes of tests on structural change. This class
of tests include, in particular, Moving Sum (MOSUM) test. In the paper from
2003 \cite{strucchange}, Zeileis et al. describe the underlying theory and
implementation of the \texttt{strucchange}
package for the R programming language, which include multiple tools for
detection of structural changes in linear regression relationships.

\subsection{The model}
\label{sec:mosum}
In \texttt{strucchange} \cite{strucchange}, a standard linear regression model is considered:
\[
y_{i}=x_{i}^{\top} \beta_{i}+u_{i} \quad(i=1, \ldots, n)
\]
where:
\begin{itemize}
\item $x_i = (1,x_{i2}, x_{i3}, ..., x_{ik})^T \in \mathbb{R}^k$
\item $y_i \in \mathbb{R}$
\item $\beta_i \in \mathbb{R}^{k} $ is the
  regression coefficient vector
\item $u_i \in \mathbb{R}$ is the residual term that is independently and identically
  distributed with mean $\mu = 0$ and variance $\sigma^2$.
\end{itemize}
We can then test for structural change by testing the null hypothesis:
\[
H_0:\quad \beta_i = \beta_0
\]
which states that there is no structural change present in the time series.
While the alternative is that the coefficient vector $\beta$ varies over time. \\\\
Let $\hat{\beta}^{(i)}$ be the ordinary least squares (OLS) estimate of the
regression coefficients based on all the observations up to $i$.
Therefore, it follows that $\hat{\beta}^{(n)}$ is the common OLS estimate in the regression model. \\\\
Let the OLS residuals be defined as:

\begin{equation} \label{eq:residuals}
\hat{u}_i = y_i - x^T\hat{\beta}^{(n)}
\end{equation}
the variance estimate would then be:
\begin{equation} \label{eq:sigma}
\hat{\sigma}^{2}=\frac{1}{n-k} \sum_{i=1}^{n} \hat{u}_{i}^{2}
\end{equation}

\subsection{Empirical fluctuation process (OLS-MOSUM)}
\label{subsec:empirical_fluctuation}
It is possible to detect structural by analyzing moving sum of residuals
\cite{strucchange}. The resulting empirical fluctuation process consists of a
sum of a fixed number of residuals in a data interval, which size is determined
by the value of the parameter $h \in (0,1)$ (bandwidth).\\\\
The OLS-based MOSUM process is therefore defined by:

\begin{equation} \label{eq:mosum}
M_{n}^{0}(t \mid h)=\frac{1}{\hat{\sigma} \sqrt{n}}
\left(\sum_{i=\left|N_{n} t\right|+1}^{\left\lfloor N_{n} t\right\rfloor+\lfloor
  n h\rfloor} \hat{u}_{i}\right) \quad(0 \leq t \leq 1-h)
\end{equation}
where $N_{n}=(n-|n h|) /(1-h)$\\\\
A key observation is that if a structural change takes place at $t_0$, the
OLS-MOSUM path would also have a strong shift at $t_0$.

\subsection{Significance testing}
\label{subsec:significance_testing}
We reject the null hypothesis of $\beta_i = \beta_0$, when the fluctuation of
the OLS-MOSUM process becomes too large. We determine whether the null
hypothesis can be rejected using a significance test.

\subsection{Steps of the algorithm}
\label{subsec:the_algorithm_steps}
\begin{enumerate}
\item \textbf{Perform the linear regression}
\item \textbf{Calculate the residuals $\hat{\mu}$, using} (\ref{eq:residuals})
\item \textbf{Calculate $\sigma$, using} (\ref{eq:sigma})
\item \textbf{Calculate the residual-based OLS-MOSUM process from}
  (\ref{eq:mosum}). \\
  Following optimization from the \texttt{strucchange} source
  code \cite{strucchange_code} can be utilized (using Python-like syntax):
  \begin{verbatim}
  nh = floor(n * h)
  process = cumsum(residuals)
  process = process[nh:] - process[:(n - nh + 1)]
  process = process / (sigma * sqrt(n))
  process = process[:n - floor(0.5 + nh / 2)]
  \end{verbatim}
\item \textbf{Calculate the statistic}
  \[
  S_{\text{MOSUM}} = \max(\|\text{process}\|)
  \]
\item \textbf{Calculate the p-value}:\\
  The p-value is calculated from the table of
  simulated asymptotic critical values of the Moving Estimate (ME) tests with
  the maximum norm \cite{moving_estimate_test}. The structure of the table can
  be seen in Table \ref{table:critvals} and there is a $4 \times 10$ table for
  each value of $k \in \{1..6\}$. Hence, we start by selecting the table that
  corresponds to our value of $k$.

  The rows in the table are
  given for $h \in \{0.05, 0.10, ..., 0.5\}$, hence we can use one-dimensional
  piecewise linear interpolation when the value of
  $h \notin \{0.05, 0.10, ..., 0.5\}$ in order to get the row of tail
  probabilities.

  Finally, we apply the linear interpolation in order to caclulate the
  probability that corresponds to the value of $S_{\text{MOSUM}}$. This is our p-value.
\item \textbf{Return the result}: \\
  If $p \leq \alpha$, we reject the null-hypothesis. In other words, we detect
  structural change in the time-series, otherwise we do not.
\end{enumerate}

\begin{table}[]
\centering
\begin{tabular}{|l|l|l|l|l|l|}
\hline
\multirow{2}{*}{k}        & \multicolumn{1}{c|}{\multirow{2}{*}{h}} & \multicolumn{4}{c|}{Tail probability}                                                                     \\ \cline{3-6} 
                          & \multicolumn{1}{c|}{}                   & 0.10                     & 0.05                     & 0.025                    & 0.01                     \\ \hline
\multirow{4}{*}{1}        & 0.05                                    & 0.7552                   & 0.8017                   & 0.8444                   & 0.8977                   \\ \cline{2-6} 
                          & 0.10                                    & 0.9809                   & 1.0483                   & 1.1119                   & 1.1888                   \\ \cline{2-6} 
                          & \multicolumn{1}{c|}{...}                & \multicolumn{1}{c|}{...} & \multicolumn{1}{c|}{...} & \multicolumn{1}{c|}{...} & \multicolumn{1}{c|}{...} \\ \cline{2-6} 
                          & 0.50                                    & 1.3560                   & 1.4938                   & 1.6166                   & 1.7663                   \\ \hline
\multirow{4}{*}{2}        & 0.05                                    & 0.7997                   & 0.8431                   & 0.8838                   & 0.9351                   \\ \cline{2-6} 
                          & 0.10                                    & 1.0448                   & 1.1067                   & 1.1634                   & 1.2388                   \\ \cline{2-6} 
                          & \multicolumn{1}{c|}{...}                & \multicolumn{1}{c|}{...} & \multicolumn{1}{c|}{...} & \multicolumn{1}{c|}{...} & \multicolumn{1}{c|}{...} \\ \cline{2-6} 
                          & 0.50                                    & 1.4884                   & 1.6125                   & 1.7266                   & 1.8639                   \\ \hline
\multicolumn{1}{|c|}{...} & \multicolumn{1}{c|}{...}                & \multicolumn{1}{c|}{...} & \multicolumn{1}{c|}{...} & \multicolumn{1}{c|}{...} & \multicolumn{1}{c|}{...} \\ \hline
\end{tabular}
\caption{An excerpt from the table of simulated asymptotic critical values of the
  ME tests with the maximum norm \cite{moving_estimate_test}.}
\label{table:critvals}
\end{table}

\subsection{Parameters and their values}
\label{subsec:mosum_parameters}
The OLS-MOSUM test uses two parameters:
\begin{itemize}
\item The bandwidth parameter $h$. The BFAST R implementation
  \cite{bfast-github}, uses $0.15$.
\item Significance level $\alpha$. The BFAST implementation
  \cite{bfast-github}, uses a standard value of $0.05$
\end{itemize}


\biblio
\end{document}
