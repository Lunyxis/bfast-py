\documentclass[main.tex]{subfiles}

\begin{document}
\chapter{Implementation}
\label{chap:implementation}
Both BFAST0n and BFAST algorithms were implemented in Python 3.8.3 using
\texttt{numpy} \cite{numpy},\texttt{statsmodels} \cite{statsmodels} and
\texttt{pandas}\cite{pandas} for all the computation and \texttt{matplotlib} for
all the plotting. The implementation is based on the original R implementation
of BFAST and BFAST0n by Verbesselt et al. \cite{bfast-github} and the
\texttt{strucchange} library by Zeileis \cite{strucchange_code}.

\section{Linear Regression}
\label{sec:linear_regression}
Linear regression is the cornerstone of BFAST and all its components such as
OLS-MOSUM test and the breakpoint estimation algorithm. Therefore it is
essential to choose a suitable implementation for this operation. There exist a
plethora of linear regression implementations for Python, but I have decided to
use the \texttt{statsmodels.regression.linear\_model.OLS} function from the
\texttt{statsmodels} library \cite{statsmodels}. It was chosen for its
stability, flexibility and detailed documentation.

\section{STL}
\label{sec:impl_stl}
For the STL decomposition algorithm, I have chosen the \texttt{tsa.seasonal.STL} function
from the \texttt{statsmodels} library. Source file \texttt{stl.py} includes a
wrapper around the \texttt{STL} method that also replaces the missing values
(NaNs) with interpolated values. It also includes a function 
\texttt{seasonal\_average} that emulates the functionality of
setting the $n_s$ parameter to ``periodic'' by taking the seasonal average of
the seasonal component of the decomposition, since \texttt{statsmodels}
implementation does not have a support for this option. For more information
about the STL parameters, please refer to Chapter \ref{chap:stl}.

\section{OLS-MOSUM Test}
\label{sec:impl_mosum}
OLS-MOSUM test is implemented in source file \texttt{efp.py}. Instance of a class \texttt{EFP()}
is created, given matrix $X$, vector $y$ and value of the bandwidth parameter
$h$ and calculates the empirical fluctuation process of type OLS-MOSUM. Then
class method \texttt{sctest} can be called in order to calculate the value of the
statistic and the p value. P value is calculated using linear interpolation from the table of critical
values in the \texttt{utils.py} by following the algorithms steps, described in
Chapter \ref{chap:mosum}.

\section{Breakpoint Estimation}
\label{sec:impl_breakpoints}
The breakpoint estimation algorithm by Bai and Perron (Chapter
\ref{chap:breakpoints}) is the most complicated step of the BFAST
algorithm and its implementation is structured in a following manner:
\begin{itemize}
\item Breakpoints are caclulated by creating an instance of a class
  \texttt{Breakpoints}, given matrix $X$, vector $y$, minimal segment length $h$
  and maximum number of breaks \texttt{breaks}.
\item The upper-triangular matrix is computed in the source file
  \texttt{ssr\_triang}. In order to speed up this step, an option for
  multi-processing is enabled and can be activated by setting \texttt{use\_mp = True}.
\item Method \texttt{breakpoints\_for\_m} estimates the number of
  breakpoints using Bayesian Information Criterion and then their location by
  calling another method \texttt{summary} that calculates the table of BIC for
  $m \in 1 \hdots m_{max}$, where $m_{max}$ is the maximum number of allowed breaks.
\end{itemize}


\section{BFAST0n and BFAST}
\label{sec:impl_bfast}
The implementation of BFAST and BFAST0n is very straight-forward and closely follows the
algorithm steps, outlined in Chapter \ref{chap:bfast} and  \ref{chap:bfast0n}
respectively. An instance of class \texttt{BFAST} calculates the time series
decomposition and the breakpoints, given matrix $X$, vector $y$, value of $h$,
significance level $\alpha$ and maximum number of breaks \texttt{breaks}.
The resulting decomposition and breaks can be accessed by returning the
object field \texttt{output}.

\biblio
\end{document}
