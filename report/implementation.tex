\documentclass[main.tex]{subfiles}

\begin{document}
\chapter{Implementation}
\label{chap:implementation}
Both BFAST0n and BFAST algorithms were implemented in Python 3.8.3 using
\texttt{numpy}, \texttt{statsmodels}\cite{statsmodels} for all the computation
and \texttt{matplotlib} for all the plotting. The implementation is based on the
original R implementation of BFAST and BFAST0n by Verbesselt et al.
\cite{bfast-github} and the \texttt{strucchange} library by Zeileis
\cite{strucchange_code}.

\section{Source Files}
\label{sec:source_files}
The source code for the implementation, described in this report, can be found
on GitHub \cite{my-github} and is structured in a following manner:
\begin{itemize}
\item \texttt{bfast0n.py}\\
  Implementation of the BFAST0N algorithm, described in Chapter \ref{chap:bfast0n}
\item \texttt{bfast.py}\\
  Implementation of the BFAST algorithm, described in Chapter \ref{chap:bfast}
\item \texttt{breakpoints.py}\\
  Implementation of the breakpoint estimation algorithm, described in Chapter \ref{chap:breakpoints}
\item \texttt{efp.py}\\
  Implementation of the empirical fluctuation processes, in particular the OLS-MOSUM test that is
  described in Chapter \ref{chap:mosum}.
\item \texttt{recresid.py}\\
  Implementation of the recursive residuals for the linear regression relationships,
  as described in Chapter \ref{chap:breakpoints}.
\item \texttt{ssr\_triang.py}\\
  Calculation of the upper-triangular matrix of sums of squared residuals for
  the breakpoint estimation algorithm as described in Chapter \ref{chap:breakpoints}.
\item \texttt{stl.py} \\
  A wrapper for the \texttt{statsmodels} \cite{statsmodels} implementation of the STL decomposition as
  described in Chapter \ref{chap:stl}.
\item \texttt{utils.py} \\
  Helper functions and lookup tables, used by other modules. 
\item \texttt{datasets.py}\\
  Collection of datasets for testing. They are taken from the R programming language standard library
  \cite{r-datasets} and
  the \texttt{strucchange} package for the R-language \cite{strucchange_code}.
\item \texttt{converter.py}\\
  Script that converts the datasets for the R programming language (\texttt{.rda}) into numpy
  data files (\texttt{.npy}). It requires a working R installation.
\end{itemize}

\section{Linear Regression}
\label{sec:linear_regression}
Linear regression the cornerstone of BFAST and all
its components such as OLS-MOSUM test and the breakpoint estimation
algorithm. Therefore it is essential to choose a suitable
implementation for this operation. There exist a plethora of linear regression
implementations for Python, but I have decided to use the \texttt{statsmodels.regression.linear\_model.OLS}
function from the \texttt{statsmodels} library \cite{statsmodels}. It was chosen
for its stability, flexibility and detailed documentation.

\section{STL}
\label{sec:impl_stl}
For the STL decomposition algorithm, I have chosen the \texttt{tsa.seasonal.STL} function
from the \texttt{statsmodels} library. Source file \texttt{stl.py} includes a
wrapper around the \texttt{STL} method that also replaces the missing values
(NaNs) with interpolated values. It also includes a function 
\texttt{seasonal\_average} that emulates the functionality of
setting the $n_s$ parameter to ``periodic'' by taking the seasonal average of
the seasonal component of the decomposition, since \texttt{statsmodels}
implementation does not have a support for this option. For more information
about the STL parameters, please refer to Chapter \ref{chap:stl}.

\section{OLS-MOSUM Test}
\label{sec:impl_mosum}



\section{Breakpoint Estimation}
\label{sec:impl_breakpoints}

\section{BFAST0n}
\label{sec:impl_bfast0n}

\section{BFAST}
\label{sec:bfast}

\biblio
\end{document}
