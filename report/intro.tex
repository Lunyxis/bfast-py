\documentclass[main.tex]{subfiles}

\begin{document}
\chapter{Introduction}
\label{chap:intro}
\section{Motivation}
\label{sec:motivation}
Climate change is the defining issue of our time, and we are at a defining
moment \cite{un}. Without drastic and swift action in the nearest future, adapting to
these changes is becoming progressively more challenging. One way, in which we
can fight this development is by monitoring and reducing global deforestation.
If the areas of deforestation are registered in a timely fashion, targeted
countermeasures could be applied, and the tide on this trend could be turned. \\\\
Meanwhile, enormous amount of satellite data is becoming available to the earth
science community. This image data can, inter alia, be utilized for the
time-series based change-detection, which is a promising tool for monitoring and
mapping of deforestation and forest degradation. Most of these approaches
operate on individual pixels. Therefore, the sheer scale of this problem (being
hundreds of billions of pixels) becomes its most significant challenge, since
there are petabytes of image data being added on a yearly basis. Without
efficient data-parallel approaches, this approach quickly becomes infeasible in
practice, even when equipped with significant computational resources and
specialized hardware. \\\\
Recently, a novel massively-parallel implementation for a
change detection method has been proposed \cite{bfast_monitor}. It combines state-of-the-art methods
for the seasonal data change-detection with modern data-parallel programming
models and yields an implementation that makes the task feasible for giant sets.
Furthermore, this implementation is feasibly executable on consumer-grade
graphical processing hardware. The authors of the paper provide a
massively-parallel implementation for the BFAST-Monitor unsupervised change
detection algorithm. 
In this project, I will research alternative approaches to
BFAST-Monitor.

\section{BFAST}
\label{sec:changes}
Remotely sensed long term data is a valuable resource that was shown to be applicable for
change detection. There are three types of changes over time that are of
interest to the earth science research:
\begin{enumerate}
\item \textbf{Seasonal change}: changes that happen withing a season (e.g. year)
\item \textbf{Gradual change}: changes that are caused by interannual climate
  variability.
\item \textbf{Abrupt change}: rapid changes that are triggered by deforestation,
  floods, fires and similar.
\end{enumerate}
In the paper from 2010, Verbesselt et al. \cite{bfast}, describe an approach for
time series change detection, which involves
detection of multiple abrupt changes in the seasonal and trend components of the time series and
characterization of such changes by their magnitude and direction. The
piecewise linear model is assumed for the seasonal and trend component and he
coefficients for the linear models are estimated using ordnary least squares
based linear regression.
This report covers the underlying theory and implementation of two algorithms based
on this approach (BFAST0n and BFAST).

\section{Contributions}
\label{sec:contributions}
The principal contributions of this project are: 

\begin{itemize}
\item I describe the underlying theory and concrete steps of the
  Seasonal and Trend decomposition using Loess (STL)
  time series decomposition algorithm, which is an essential part of BFAST.
  In particular, the description and values for all algorithm parameters are
  covered (Chapter \ref{chap:stl}).
\item Describe the underlying theory and link it to the concrete steps of
  the Ordinary Least Squares Moving Sum (OLS-MOSUM) test from the
  \texttt{strucchange} package \cite{strucchange} for the R programming
  language by Achim Zeileis. I also provide examples to illustrate the
  operation of the non-trivial parts of the algorithm. OLS-MOSUM test is another
  important building block of BFAST (Chapter \ref{chap:mosum}).
\item Describe the theory and algorithm steps of the nontrivial multiple
  breakpoint estimation algorithm by Bai and Perron, based on the paper from 2003 \cite{bai_perron}
  and provide illustrative examples. This dynamic programming-based algorithm is
  an integral component of BFAST (Chapter \ref{chap:breakpoints}).
\item Describe the theory and steps of the BFAST and BFAST0n structural change
  detection algorithms by Jan Verbesselt et. al. \cite{bfast} (Chapters
  \ref{chap:bfast} and \ref{chap:bfast0n}).
\item Provide a prototype implementation of BFAST0n and BFAST in a
  general-purpose programming language (Python) (Chapter \ref{chap:implementation}).
\item Validate the Python implementation using artificial and real-life small
  datasets from the original paper by Verbesselt et al. \cite{bfast} (Chapter \ref{chap:validation}).
\end{itemize}

\biblio
\end{document}
