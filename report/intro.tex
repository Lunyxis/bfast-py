\section{Introduction}
\label{sec:intro}
Climate change is the defining issue of our time, and we are at a defining
moment \cite{un}. Without drastic and swift action in the nearest future, adapting to
these changes is becoming progressively more challenging. One way, in which we
can fight this development is by monitoring and reducing global deforestation.
If the areas of deforestation are registered in a timely fashion, targeted
countermeasures could be applied, and the tide on this trend could be turned. \\\\
Meanwhile, enormous amount of satellite data is becoming available to the earth
science community. This image data can, inter alia, be utilized for the
time-series based change-detection, which is a promising tool for monitoring and
mapping of deforestation and forest degradation. Most of these approaches
operate on individual pixels. Therefore, the sheer scale of this problem (being
hundreds of billions of pixels) becomes its most significant challenge, since
there are petabytes of image data being added on a yearly basis. Without
efficient data-parallel approaches, this approach quickly becomes infeasible in
practice, even when equipped with significant computational resources and
specialized hardware. \\\\
Recently, a novel massively-parallel implementation for a
change detection method has been proposed. It combines state-of-the-art methods
for the seasonal data change-detection with modern data-parallel programming
models and yields an implementation that makes the task feasible for giant sets.
Furthermore, this implementation is feasibly executable on consumer-grade
graphical processing hardware. The authors of the paper provide a
massively-parallel implementation for the BFAST-Monitor unsupervised change
detection algorithm. \\\\
In this project, I will research alternative approaches to
the BFAST-Monitor by surveying the relevant literature. I would then investigate
and analyze the strengths and weaknesses of the different methods. Finally, one
or more algorithms would be implemented using a general-purpose programming
language (Python). The results of this project would form a solid foundation for
the future work, where the parallelized implementation would be developed and
could potentially serve as an alternative to the existing solution.

%% Hence the main
%% contributions of this project are:

%% \begin{itemize}
%% \item Perform a literature survey on the topic of
%% change-detection algorithms.
%% \item Compare the existing state-of-the art
%% change-detection algorithms, investigate their strengths and weaknesses.
%% \item Provide a prototype implementation of one or more algorithms in a
%% general-purpose programming language.
%% \end{itemize}

\subsection{Trend and Seasonal Changes in Time Series}
\label{subsec:trend_and_seasonal_changes_in_time_series}
