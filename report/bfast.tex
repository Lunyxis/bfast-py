\documentclass[main.tex]{subfiles}

\begin{document}
\chapter{BFAST}
\label{chap:bfast}
In paper from 2010, Verbesselt et. al \cite{bfast} outline a generic change detection approach
that combines iterative decomposition into trend, seasonal and remainder components and detection
and characterizing of breakpoints within a time series. 

\section{The model}
\label{sec:bfast_the_model}
For BFAST, we consider a following data model:
\[
Y_t = T_t + S_t + e_t, \quad t = 1,...,n
\]
where:
\begin{itemize}
\item $Y_t \in \mathbb{R}$ is the observation at time $t$
\item $T_t \in \mathbb{R}$ is the trend component at time $t$
\item $S_t \in \mathbb{R}$ is the seasonal component at time $t$
\item $e_t \in \mathbb{R}$ is the remainder component at time $t$
\item $n$ is the number of observations in time series
\end{itemize}
For more information about the time series components, please refer to Chapter
\ref{chap:stl}.
\subsection{The Trend Component}
\label{subsec:trend}
We assume that $T_t$ is piecewise linear and has breakpoints $t_1^*,\hdots, t_m^*$,
where $m$ is the number of breakpoints in the trend component and set $t_0^* = 0$.
Then, the trend component can be described as
\[
T_t = \alpha_j + \beta_j t \quad \text{for}\quad t^*_{j-1}<t\leq t_j^*
\]
where $j$ is the number of the next breakpoint, i.e. $j = 1,...,m$. One way, in
which we can characterize the abrupt changes in the trend component is by
calculating the magnitude of the change:
\[
\operatorname{Magnitude}(j) = T_{j-1} - T_{j} = \alpha_{j-1} + \beta_{j-1} t -
(\alpha_j + \beta_j t) = (\alpha_{j-1} - \alpha_j) + (\beta_{j-1} - \beta_j)t
\]
where $j = 1,...,m$.

\subsection{The Seasonal Component - Dummy Model}
\label{subsec:seasonal_dummy}
The breakpoints in the seasonal component can occur at different times than the
breaks in the trend component. Let
$t_1^{\#},\hdots, t_p^{\#}$ be the breakpoints in the seasonal component,
where $p$ is the number of breakpoints and $t_0^{\#} = 0$.\\\\
For $t_{j-1}^{\#} < t \leq t_j^{\#}$, the seasonal term can be expressed as:
\[
S_t = \sum_{i=1}^{s-1}\gamma_{i,j}(d_{t, i} - d_{t, 0})
\]
where
\begin{itemize}
\item $s$ is the period of seasonality
\item $\gamma_{i,j}$  is the influence of season $i$ that we are going to estimate
with linear regression. 
\item $d_{t,i}$ is the dummy variable \cite{makridakis}, for which it holds that
  $d_{t,i} = 1$ when $t$ is in season i and 0 otherwise. Then for season 0, we
  have that $d_{t, i} - d_{t, 0} = -1$ and $d_{t, i} - d_{t, 0} = 1$ otherwise.
\end{itemize}
An important observation is that the sum of $S_t$ across $s-1$ successive elements is zero.
The model is build this way in order to avoid breakpoints in
trend being caused by seasonal breaks. \\\\
\subsubsection{Breakpoint Detection and Estimation}
When detecting breakpoints in the seasonal component, we apply the breakpoint
detection algorithm to the model (in matrix form):
\[
X_d =
\begin{bmatrix}
 1 & 0 & 0 & \hdots & 0\\
 0 & 1 & 0 & \hdots & 0 \\
 0 & 0 & 1 & \hdots & 0 \\
\vdots & \vdots & \vdots & \ddots & \vdots \\
 0 & 0 & 0 & \hdots & 1 \\
 -1 & -1 & -1 & \hdots & -1 \\
 1 & 0 & 0 & \hdots & 0\\
 0 & 1 & 0 & \hdots & 0 \\
 0 & 0 & 1 & \hdots & 0 \\
\vdots & \vdots & \vdots & \ddots & \vdots \\
 0 & 0 & 0 & \hdots & 1 \\
 -1 & -1 & -1 & \hdots & -1 \\
 1 & 0 & 0 & \hdots & 0\\
\vdots & \vdots & \vdots & \vdots & \vdots \\
\end{bmatrix}
\]
where $X_\text{d}$ has $s-1$ columns and $n$ rows.\\\\
We then use the same model for the breakpoint estimation algorithm by Bai and Perron.
The estimated breakpoints
$t_1^{\#},\hdots, t_p^{\#}$ can then be used to apply the linear regression to estimate
the vector of seasonal coefficients $\gamma$ ($\gamma_{i,j}$ for
$j = 1, \hdots, p$ and $i = 1,\hdots s-1$).
\[
Y - T = X_{part} \gamma
\]
where $Y$ is the vector of observations, $T$ is the trend component and
$X_{part}$ is matrix $X_d$ that was partitioned according to the
estimated breakpoints $t_1^{\#},\hdots, t_p^{\#}$. Consider an example with a
single breakpoint, where $n = 12$, $s=4$ and $t_1^{\#} = 5$. Then:
\[
X_{part} =
\begin{bmatrix}
  1 & 0 & 0 & 0 & 0 & 0\\
  0 & 1 & 0 & 0 & 0 & 0\\
  0 & 0 & 1 & 0 & 0 & 0\\
  -1 & -1 & -1 & 0 & 0 & 0\\
  1 & 0 & 0 & 0 & 0 & 0\\
  0 & 0 & 0 & 0 & 1 & 0\\
  0 & 0 & 0 & 0 & 0 & 1\\
  0 & 0 & 0 & -1 & -1 & -1\\
  0 & 0 & 0 & 1 & 0 & 0\\
  0 & 0 & 0 & 0 & 1 & 0\\
  0 & 0 & 0 & 0 & 0 & 1\\
  0 & 0 & 0 & -1 & -1 & -1
\end{bmatrix}
\]
Linear regression is applied to estimate the values:
$\hat{\gamma}_{1,1}, \hat{\gamma}_{1,2}, \hat{\gamma}_{1,3}, \hat{\gamma}_{2,1},
\hat{\gamma}_{2,2}$ and $\hat{\gamma}_{2,3}$.\\\\
The seasonal component can then be estimated:
$\hat{S}_t = \sum_{i=1}^{s-1}\hat{\gamma}_{i,j}(d_{t,i} - d_{t,0})$, or in
matrix form $\hat{S} = X_{\text{part}} \gamma$

\subsection{The Seasonal Component - Harmonic Model}
\label{subsec:seasonal_harmonic}
Alternatively, we can use the Harmonic model that is also used by the
BFAST0n algorithm, for $K=3$. For more information, please refer to Chapter
\ref{chap:bfast0n}. We have:
\[
X_{h} =
\begin{bmatrix}
  \sin\nicefrac{2 \pi}{f} & \cos\nicefrac{2 \pi}{f} & \sin\nicefrac{4 \pi}{f} & \cos\nicefrac{4
    \pi}{f} &  \sin\nicefrac{6 \pi}{f} & \cos\nicefrac{6 \pi}{f} \\
  \sin\nicefrac{4 \pi}{f} & \cos\nicefrac{4 \pi}{f} & \sin\nicefrac{8 \pi}{f} & \cos\nicefrac{8
    \pi}{f} &  \sin\nicefrac{12 \pi}{f} & \cos\nicefrac{12 \pi}{f} \\
  \vdots & \vdots  & \vdots & \vdots & \vdots & \vdots \\
  \sin\nicefrac{2 \pi n}{f} & \cos\nicefrac{2 \pi n}{f} & \sin\nicefrac{4 \pi n}{f} &
  \cos\nicefrac{4 \pi n}{f} &  \sin\nicefrac{6 \pi n}{f} & \cos\nicefrac{6 \pi n}{f}
\end{bmatrix}
\]
For the application of linear regression $X_h$ would be partitioned the same way
as $X_d$ (dummy model), resulting in matrix $X_{\text{part}}$ with $n$ rows and
$6p$ columns, where $p$ is the number of breakpoints in the seasonal component.
The estimate for the seasonal component can be rebuilt using:
\[
\hat{S} = X_{\text{part}} \omega
\]
where $\omega$ is a $(1 \times 6p)$ vector
$(\gamma_{1, 1}, \; \theta_{1, 1},\;  \gamma_{1, 2},\;  \theta_{1, 2},\;
\hdots,\;  \gamma_{p, 3},\;  \theta_{p, 3})^{\top}$

\section{Steps of the Algorithm}
\label{sec:bfast_algorithm_steps}
BFAST operates in a following way:
\begin{enumerate}
\item \textbf{Estimate $\hat{S}_t$ using STL} and finding the average of all seasons, by
  setting $n_s =$ ``harmonic''. For more information about STL parameters,
  please refer to Chapter \ref{chap:stl}.
\item Then iterate:
  \begin{enumerate}[1)]
    \item \textbf{Calculate the deasonalized time series: $V_t = Y_t - \hat{S}_t$}
    \item \textbf{Apply the OLS-MOSUM test} (Chapter \ref{chap:mosum}) to $x=\operatorname{ti}$ and
      $y = V_t$. Where $\operatorname{ti}$ denotes the values of the time interval of $Y_t$.
      If the returned p-value is lower than the significance level $\alpha$,
      \textbf{estimate the number and position of the trend components} using the
      breakpoint estimation algorithm by Bai and Perron (Chapter
      \ref{chap:breakpoints}) from $ti$ and $V_t$.
    \item \textbf{Compute the trend coefficients} $\alpha_j$ and $\beta_j$ for
      $j = 1, \hdots, m$ using linear regression. Set the trend estimate
      $\hat{T}_t = \hat{\alpha}_j + \hat{\beta}_j t$ for
      $t = t^*_{j-1} + 1, \hdots, t^*_j$.
    \item \textbf{Calculate the detrended time series}: $W_t = Y_t - \hat{T}_t$
    \item If the OLS-MOSUM test, applied to $y=W_t$ and
      the chosen seasonal model signifies that the breakpoints are present in the
      seasonal data, \textbf{estimate the number and position of the
      breakpoints in the trend component} using the breakpoint estimation
      algorithm, as described in Section \ref{subsec:seasonal_dummy} and
      \ref{subsec:seasonal_harmonic}.
    \item \textbf{Compute the coefficients for the seasonal component
      and reconstruct $\hat{S}_t$} from the chosen seasonal model and seasonal
      coefficient, as described in Section \ref{subsec:seasonal_dummy} and
      \ref{subsec:seasonal_harmonic}.
  \end{enumerate}
  The iteration is stopped when the number and position of the breakpoints do
  not change during the iteration, or when we reach the maximum allowed number of iterations.
\item \textbf{Return} $\hat{T}_t$, $\hat{S}_t$ the number and position of trend and
  seasonal breakpoints and the maximum breakpoint magnitude.
\end{enumerate}

\section{Parameters and Their Values}
\label{sec:bfast_params}
BFAST algorithm has following paramers
\begin{itemize}
\item $s$. Period of seasonality. Specific for each dataset.
\item $0<h<1$. Minimal segment length for the Breakpoint estimation and
  bandwidth parameter for the OLS-MOSUM test. For more information please refer
  to Chapters \ref{chap:mosum} and \ref{chap:breakpoints} respectively.
  The default value is $0.15$.
\item \texttt{seasonal}. Type of model to be fitted. Should be either
  ``harmonic'', ``dummy'' or ``none''. The later means that no seasonal model
  would be fitted, i.e. $S_t=0$.
\item \texttt{max\_iter}. Maximal number of iterations of the main loop. The
  default value is $10$. 
\item \texttt{max\_breaks}. Upper limit on the number of breakpoints that could be computed by
  the breakpoint estimation algorithm. The default value is ``None'', i.e. no upper limit.
  For more information, please refer to Chapter \ref{chap:breakpoints}.
\item $\alpha$. The significance level for the OLS-MOSUM test. If the returned
  p value is below that threshold - we consider that there is structural change
  and proceed with the linear regression. Higher values of $\alpha$ make BFAST
  more sensitive to structural change. For more information about the confidence
  intervals, please refer to the Chapter \ref{chap:breakpoints}. The default
  value is $0.05$.
\end{itemize}

\biblio
\end{document}
