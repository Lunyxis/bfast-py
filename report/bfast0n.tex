\documentclass[main.tex]{subfiles}

\begin{document}
\chapter{BFAST0n}
\label{chap:bfast0n}
BFAST0N is the light-weight variation on the BFAST approach (Chapter
\ref{chap:bfast}). The idea behind this method is to apply the breakpoint
estimation algorithm by Bai and Perron \cite{bai_perron},  which is described in Chapter
\ref{chap:breakpoints},  to a pre-processed dataset,

\section{Dataset Pre-processing}
\label{sec:harmonic_model}
Let $Y_t$ be the observed data for $t$ in $(1, \hdots, n)$.
\subsection{STL}
\label{subsec:stl}
Before any other pre-processing takes place, STL time series decomposition algorithm is
used for trend- and/or season-adjustment. For more information about the STL
algorithm, please refer to Chapter {\ref{chap:stl}}.

\subsection{Harmonic model}
\label{subsec:harmonic_model}
The second step of dataset pre-processing is the application of the harmonic
model, that was described in paper by Verbesselt et. al from 2010.
\cite{bfast1}.
\[
S_{t} =
\sum_{k=1}^{K}\left[\gamma_{j, k} \sin \left(\frac{2 \pi k t}{f}\right)+
  \theta_{j, k} \cos \left(\frac{2 \pi k t}{f}\right)\right]\\
\]
where
\begin{itemize}
\item $K$ is the harmonic term, i.e. how many of pairs of sin + cos terms we use,
\item $\gamma_{j, k}$ and $\theta_{j, k}$ are the seasonal coefficients to be
  estimated by linear regression in the breakpoint estimation algorithm.
\item $t$ is the observation time.
\item $f$ frequency of the dataset
\end{itemize}
In order to apply the model to the time-series s.t. it could be used by the
breakpoint estimation algorithm, we need to compute the matrix form of the
harmonic model (for $K=3$):
\[
X_{h} =
\begin{bmatrix}
  \sin\nicefrac{2 \pi}{f} & \cos\nicefrac{2 \pi}{f} & \sin\nicefrac{4 \pi}{f} & \cos\nicefrac{4
    \pi}{f} &  \sin\nicefrac{6 \pi}{f} & \cos\nicefrac{6 \pi}{f} \\
  \sin\nicefrac{4 \pi}{f} & \cos\nicefrac{4 \pi}{f} & \sin\nicefrac{8 \pi}{f} & \cos\nicefrac{8
    \pi}{f} &  \sin\nicefrac{12 \pi}{f} & \cos\nicefrac{12 \pi}{f} \\
  \vdots & \vdots  & \vdots & \vdots & \vdots & \vdots \\
  \sin\nicefrac{2 \pi n}{f} & \cos\nicefrac{2 \pi n}{f} & \sin\nicefrac{4 \pi n}{f} &
  \cos\nicefrac{4 \pi n}{f} &  \sin\nicefrac{6 \pi n}{f} & \cos\nicefrac{6 \pi n}{f}
\end{bmatrix}
\]
$X_h$ has $n$ rows and $2K$ columns.

\section{Steps of the Algorithm}
\label{sec:bfast0n_algorithm_steps}
BFAST0n consists of following steps:
\begin{enumerate}
\item \textbf{Apply STL to the observation matrix $Y$} in order to get the adjusted
  time series $\tilde{Y}$
\item \textbf{Apply the harmonic model to $\tilde{Y}$}
\item \textbf{Use the breakpoint estimation algorithm} (Chapter \ref{chap:breakpoints})
  to $x=X_h$ and $y=\tilde{Y}$
\item \textbf{Return the number and placement of the breakpoints}
\end{enumerate}

\section{Parameters and Their Values}
\label{sec:bfast0n_params}
BFAST0n has following parameters:
\begin{itemize}
\item $f$. Frequency of the time series. Specific for each dataset.
\item \texttt{stl}. Which type of STL-based pre-processing to use.
  Allowed values:
  \begin{itemize}
  \item ``none'' - the default value. No STL adjustment would be performed
  \item ``trend'' - Trend component is removed from the input time series
  \item ``seasonal'' - Seasonal component is removed from the input time series
  \item ``both'' - Seasonal component is removed from the input time series
  \end{itemize}
\item $0 \leq K \leq 3$. The integer value of the harmonic term, if no value
  provided, a value of $3$ is used.
\end{itemize}
\biblio
\end{document}
