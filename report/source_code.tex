\documentclass[main.tex]{subfiles}

\begin{document}
\chapter{Source code overview}
\label{a_chap:source_code_overview}
The source code for implementation, described in this report, can be found on GitHub \cite{my-github}.
\section{Source Files}
\label{sec:source_files}
The code is structured in a following manner:
\begin{itemize}
\item \texttt{bfast0n.py}\\
  Implementation of the BFAST0N algorithm, described in Chapter \ref{chap:bfast0n}
\item \texttt{bfast.py}\\
  Implementation of the BFAST algorithm, described in Chapter \ref{chap:bfast}
\item \texttt{breakpoints.py}\\
  Implementation of the breakpoint estimation algorithm, described in Chapter \ref{chap:breakpoints}
\item \texttt{efp.py}\\
  Implementation of the empirical fluctuation processes, in particular the OLS-MOSUM test that is
  described in Chapter \ref{chap:mosum}.
\item \texttt{recresid.py}\\
  Implementation of the recursive residuals for the linear regression relationships,
  as described in Chapter \ref{chap:breakpoints}.
\item \texttt{stl.py} \\
  A wrapper for the \texttt{statsmodels} \cite{statsmodels} implementation of the STL decomposition as
  .described in Chapter \ref{chap:stl}. The implementation
\item \texttt{datasets.py}\\
  Collection of datasets for testing. They are taken from the R programming language standard library
  \cite{r-datasets} and
  the \texttt{strucchange} package for the R-language \cite{strucchange_code}.
\end{itemize}

\section{Tests}
\label{sec:source_files}
Tests for each file are contained withing that test file. In order to run the test, run:
\begin{verbatim}
python file.py
\end{verbatim}
In order to see the debug information, run:
\begin{verbatim}
python file.py --log=DEBUG
\end{verbatim}

\biblio
\end{document}
