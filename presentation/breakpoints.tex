\documentclass[presentation.tex]{subfiles}

\begin{document}

\begin{frame}
\frametitle{Breakpoint Estimation - Intro}
\begin{itemize}
  \item Described in the paper by Bai and Perron from 2003
  \item Estimate the number and position of breakpoints in a time series using
    a dynamic programming algorithm and Bayesian Information Criterion (BIC)
\end{itemize}
\end{frame}


\begin{frame}
\frametitle{Breakpoint Estimation - The Model}
\begin{itemize}
  \item 
We assume a pure structural change model for $m$ breaks ($m+1$ segments):
\[
y_{t} = x_{t}^{\top} \beta_{j}+u_{t} \quad t=T_{j-1}+1, \ldots, T_{j}
\]
for $j = 1,\hdots , m+1$, and where:
\vspace{2.5mm}
\begin{itemize}
\item $x_t \in \mathbb{R}^q$ is the value of the independent variable at time
  $t = 1,\hdots , T$
\item $y_{t} \in \mathbb{R}$ is the observation at time $t$
\item $\beta_j: \; (j=1, \hdots ,m+1)$ is the vector of coefficients for the segment $j$
\item $u_{t} \in \mathbb{R}$ is the disturbance (error) at time $t$
\item $(T_1, \hdots ,T_m)$ are the unknown indices of the $m$ breakpoints.
 We additionally set $T_0 = 0$ and $T_{m+1} = T$
\end{itemize}
\item 
In other words, we split the time series into $m+1$ segments (of potentially
different sizes) and perform linear
regression independently for each segment, where for segment $i$, we estimate
a coefficient vector $\beta_i$.
Hence, we find the estimate of the coefficient vector ($\hat{\beta}$) for each
m-partition $(T_1, ..., T_m)$ by minimizing the sum of squared residuals:
\[
\sum_{i=1}^{m+1}\sum_{i=T_{i-1} + 1}^{T_i} \left[ y_t - x_t^{\top}\beta_i \right]^2
\]
\end{itemize}
\end{frame}

\begin{frame}
  \frametitle{}
\begin{itemize}
\item
Let $\{T_j\}$ denote an m-partition $(T_1, ..., T_m)$ and $S_T(T_1, ..., T_m)$
denote the resulting sum of squared residuals. Since we can estimate the
coefficient vector for each partition, we can minimize the sum of
squared residuals by finding the optimal position for the breakpoints
\[
(\hat{T}_1, ..., \hat{T}_m) = \operatorname{argmin}_{T_{1}, \ldots, T_{m}}
S_{T}\left(T_{1}, \ldots, T_{m}\right)
\]
\item
There is a finite number of possible breakpoints, and only a subset of
possible partitions is feasible. There are $T(T+1)/2$ (sum of integers from 1 to T)
possible segments that can be chosen. This can be demonstrated by building a
matrix of possible segments, with starting date on the y-axis and terminal date
on the y-axis. A length of a segment is positive, hence we can eliminate
one half of the potential segments. There are further reductions
that can be made that reduce the number of feasible segments to
$T(T+1)/2 - \left(T(h-1)-m h(h-1)-(h-1)^{2}-h(h-1) / 2\right)$, where $h$ is the minimal
segment length.
\end{itemize}
\end{frame}

\begin{frame}[fragile]
  \frametitle{Upper-diagonal Matrix of Sums of Squared Residuals}
\begin{table}[]
\begin{tabular}{lllllllllll}
                                   & \multicolumn{10}{c}{stop}                                                                                                                                                                                                                                                 \\ \cline{2-11} 
\multicolumn{1}{l|}{}              & \multicolumn{1}{l|}{}  & \multicolumn{1}{l|}{1}    & \multicolumn{1}{l|}{2}    & \multicolumn{1}{l|}{3}    & \multicolumn{1}{l|}{4}    & \multicolumn{1}{l|}{5}    & \multicolumn{1}{l|}{6}    & \multicolumn{1}{l|}{7}    & \multicolumn{1}{l|}{8}    & \multicolumn{1}{l|}{9}    \\ \cline{2-11} 
\multicolumn{1}{l|}{}              & \multicolumn{1}{l|}{1} & \multicolumn{1}{l|}{$\operatorname{nf}_1$} & \multicolumn{1}{l|}{$\operatorname{nf}_1$} & \multicolumn{1}{l|}{$\operatorname{nf}_1$} & \multicolumn{1}{l|}{f}   & \multicolumn{1}{l|}{f}   & \multicolumn{1}{l|}{f}   & \multicolumn{1}{l|}{$\operatorname{nf}_2$} & \multicolumn{1}{l|}{$\operatorname{nf}_2$} & \multicolumn{1}{l|}{$\operatorname{nf}_2$} \\ \cline{2-11} 
\multicolumn{1}{l|}{}              & \multicolumn{1}{l|}{2} & \multicolumn{1}{l|}{}     & \multicolumn{1}{l|}{$\operatorname{nf}_1$} & \multicolumn{1}{l|}{$\operatorname{nf}_1$} & \multicolumn{1}{l|}{$\operatorname{nf}_1$} & \multicolumn{1}{l|}{$\operatorname{nf}_3$} & \multicolumn{1}{l|}{$\operatorname{nf}_3$} & \multicolumn{1}{l|}{$\operatorname{nf}_2$} & \multicolumn{1}{l|}{$\operatorname{nf}_2$} & \multicolumn{1}{l|}{$\operatorname{nf}_2$} \\ \cline{2-11} 
\multicolumn{1}{l|}{}              & \multicolumn{1}{l|}{3} & \multicolumn{1}{l|}{}     & \multicolumn{1}{l|}{}     & \multicolumn{1}{l|}{$\operatorname{nf}_1$} & \multicolumn{1}{l|}{$\operatorname{nf}_1$} & \multicolumn{1}{l|}{$\operatorname{nf}_1$} & \multicolumn{1}{l|}{$\operatorname{nf}_3$} & \multicolumn{1}{l|}{$\operatorname{nf}_2$} & \multicolumn{1}{l|}{$\operatorname{nf}_2$} & \multicolumn{1}{l|}{$\operatorname{nf}_2$} \\ \cline{2-11} 
\multicolumn{1}{l|}{start} & \multicolumn{1}{l|}{4} & \multicolumn{1}{l|}{}     & \multicolumn{1}{l|}{}     & \multicolumn{1}{l|}{}     & \multicolumn{1}{l|}{$\operatorname{nf}_1$} & \multicolumn{1}{l|}{$\operatorname{nf}_1$} & \multicolumn{1}{l|}{$\operatorname{nf}_1$} & \multicolumn{1}{l|}{f}   & \multicolumn{1}{l|}{f}   & \multicolumn{1}{l|}{f}   \\ \cline{2-11} 
\multicolumn{1}{l|}{}              & \multicolumn{1}{l|}{5} & \multicolumn{1}{l|}{}     & \multicolumn{1}{l|}{}     & \multicolumn{1}{l|}{}     & \multicolumn{1}{l|}{}     & \multicolumn{1}{l|}{$\operatorname{nf}_1$} & \multicolumn{1}{l|}{$\operatorname{nf}_1$} & \multicolumn{1}{l|}{$\operatorname{nf}_1$} & \multicolumn{1}{l|}{f}   & \multicolumn{1}{l|}{f}   \\ \cline{2-11} 
\multicolumn{1}{l|}{}              & \multicolumn{1}{l|}{6} & \multicolumn{1}{l|}{}     & \multicolumn{1}{l|}{}     & \multicolumn{1}{l|}{}     & \multicolumn{1}{l|}{}     & \multicolumn{1}{l|}{}     & \multicolumn{1}{l|}{$\operatorname{nf}_1$} & \multicolumn{1}{l|}{$\operatorname{nf}_1$} & \multicolumn{1}{l|}{$\operatorname{nf}_1$} & \multicolumn{1}{l|}{f}   \\ \cline{2-11} 
\multicolumn{1}{l|}{}              & \multicolumn{1}{l|}{7} & \multicolumn{1}{l|}{}     & \multicolumn{1}{l|}{}     & \multicolumn{1}{l|}{}     & \multicolumn{1}{l|}{}     & \multicolumn{1}{l|}{}     & \multicolumn{1}{l|}{}     & \multicolumn{1}{l|}{$\operatorname{nf}_1$} & \multicolumn{1}{l|}{$\operatorname{nf}_1$} & \multicolumn{1}{l|}{$\operatorname{nf}_1$} \\ \cline{2-11} 
\multicolumn{1}{l|}{}              & \multicolumn{1}{l|}{8} & \multicolumn{1}{l|}{}     & \multicolumn{1}{l|}{}     & \multicolumn{1}{l|}{}     & \multicolumn{1}{l|}{}     & \multicolumn{1}{l|}{}     & \multicolumn{1}{l|}{}     & \multicolumn{1}{l|}{}     & \multicolumn{1}{l|}{$\operatorname{nf}_1$} & \multicolumn{1}{l|}{$\operatorname{nf}_1$} \\ \cline{2-11} 
\multicolumn{1}{l|}{}              & \multicolumn{1}{l|}{9} & \multicolumn{1}{l|}{}     & \multicolumn{1}{l|}{}     & \multicolumn{1}{l|}{}     & \multicolumn{1}{l|}{}     & \multicolumn{1}{l|}{}     & \multicolumn{1}{l|}{}     & \multicolumn{1}{l|}{}     & \multicolumn{1}{l|}{}     & \multicolumn{1}{l|}{$\operatorname{nf}_1$} \\ \cline{2-11} 
\end{tabular}
\caption{An example of an upper-triangular matrix of sums of squared residuals for $T =
  9$, $h=3$, $m=2$. We must compute sum of squared residuals for all feasible
  segments (f).
}
\end{table}
\end{frame}



%% \begin{frame}
%%   \frametitle{Breakpoint Estimation - The Model Continued}
%%   \begin{itemize}
%%     \item 
%% This linear regression system can also be expressed in matrix form:
%% \[
%% Y = \overline{X}\beta + U
%% \]
%% where:
%% \[
%% Y =
%% \begin{bmatrix}
%% y_1 \\
%% y_2 \\
%% \vdots \\
%% y_T
%% \end{bmatrix}
%% \quad
%% \beta =
%% \begin{bmatrix}
%% \vert & \vert &  & \vert \\
%% \beta_{1} & \beta_{2}  & \hdots & \beta_{m+1}\\
%% \vert & \vert &  & \vert\\
%% \end{bmatrix}
%% \quad
%% U =
%% \begin{bmatrix}
%% u_1 \\
%% u_2 \\
%% \vdots \\
%% u_T
%% \end{bmatrix}
%% \]
%% and $\overline{X}$ is a block matrix that
%% diagonally partitions $X$ at breaking points $(T_1,..., T_m)$:
%% \[
%% \overline{X} =
%% \begin{bmatrix}
%% X_1 & 0 & \hdots & 0\\
%%  0 & X_2 & \hdots & 0 \\
%% \vdots & \vdots & \ddots & \vdots \\
%%  0 & 0 & \hdots & X_{m+1} 
%% \end{bmatrix}
%% \quad
%% X_i = 
%% \begin{bmatrix}
%%   \text{---} \hspace{-0.2cm} & x_{T_{i-1} + 1}^{\top} & \hspace{-0.2cm}\text{---} \\
%%   \text{---} \hspace{-0.2cm} & x_{T_{i-1} + 2}^{\top} & \hspace{-0.2cm}\text{---} \\
%%   & \vdots & \\ 
%%  \text{---} \hspace{-0.2cm} & x_{T_i}^{\top} & \hspace{-0.2cm}\text{---}  \\
%% \end{bmatrix}
%% \]
%% \end{itemize}  
%% \end{frame}

\end{document}
